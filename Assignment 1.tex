\documentclass[11pt,a4paper,final]{report}
\usepackage[utf8]{inputenc}
\usepackage{amsmath}
\usepackage{amsfonts}
\usepackage{amssymb}


\begin{document}

%\selectlanguage{english} %%% remove comment delimiter ('%') and select language if required


\noindent \textbf{ASSESSING THE GENERAL PERSPECTIVE OF WHO A BEST FRIEND IS ACCORDING TO A SAMPLE OF COMPUTER SCIENCE STUDENTS BY GENDA IN ORDER TO IMPROVE THEIR SOCIAL ELECTRONIC CONNECTION}

\noindent \textbf{}

\noindent \textbf{BY: }KYESWA LUTIMBA IVAN\textbf{}

\noindent \textbf{STUDENT NUMBER: }216001516

\noindent \textbf{REGISTRATION NUMBER: }16/U/512

\noindent\\

\noindent \textbf{BACKGROUND OF TOPIC}\\

\noindent The aim of this report is to provide knowledge learnt from a research made about the above statement. We all have a best friend well at least most of us. It is that someone who seems to rank highest among your friends. Very often times we don't seat done to rank our friends to find out who our best friends are. It is something so natural. I believe that there is a pattern that leads to this natural conclusion and it is what I have investigated. The results of this research can tell us a lot about why things turn out the way they do.\\

\noindent 

\noindent \textbf{INTRODUCTION:}
\\
\noindent A friend is anybody with whom you are satisfied with what you know about him/her and can easily interact with him/her. Now, a best friend as the name suggests is simple the best of your friends. Not necessarily that he/she is better than all your other friends in making you happy but one who you are most satisfied with what you know about him/her and  are find it easier to interact with him/her.

\noindent In this research I wanted to find out what a sample of computer science year two students consider a best friend to be. By making this research I was able to come up with several conclusions as will be stated later wards.

\noindent \\

\noindent \textbf{METHOD OF DATA COLLECTION:}

\noindent The method used to collect the data was simply an oral interview following a six\eqref{GrindEQ__6_} question Questionnaire. A sample of ten \eqref{GrindEQ__10_} students was interviewed five \eqref{GrindEQ__5_} for each sex. All the participants were interviewed privately with a high level of obstruction so that the information collected would be highly reliable.

\noindent 

\noindent Bellow is the questionnaire:

\noindent 

\begin{enumerate}
\item  \textbf{Do you have a best friend? If no, "why not?" and if yes continue?}

\item \textbf{ What is the gender of your best friend?}

\item \textbf{ Which of these best describes your best friend's age?}

\item \textbf{ younger peer older}

\item \textbf{ Why is that person your best friend?}

\item \textbf{ How long have you been friends with you your best friend?}

\item \textbf{ Can your best friend be one of your relatives?}
\end{enumerate}

\noindent \textbf{}

\noindent \textbf{RESULTS}

\noindent The results are tabulated bellow as fractional representation of conceptual agreement amongst participants.\\

\noindent 

\begin{tabular}{|p{1.9in}|p{1.1in}|p{1.0in}|} \hline 
\textbf{Concept} & \textbf{male} & \textbf{female} \\ \hline 
I have a best friend. & 4 of 5 & 4 of 5 \\ \hline 
I have a best friend of the same sex. & 2 of 4 & 4 of 4 \\ \hline 
My best friend in my peer & 2 of 4 & 2 of 4 \\ \hline 
My best friend is younger & 0 of 4 & 1 of 4 \\ \hline 
My best friend is older & 2 of 4 & 1 of 4 \\ \hline 
My best friend can be my relative & 3 of 4 & 3 of 4 \\ \hline 
\end{tabular}\\



\noindent 

\noindent \textbf{Other concepts}

\begin{enumerate}
\item \textbf{ }Both male and female participants with best friends have spent a year or more with their best friend with the male range at 11/2 - 10 years while female range is 1 - 12 years. 

\item  The participants both male and female who didn't have best friends reasoned that they saw nothing so special about any of their friends in order to consider them best friends.

\item  The reasons for being a best friend where given as listed bellow;
\begin{enumerate}

\item  We understand each other.

\item  We have the same interests and understand each other.

\item  He knows me

\item  We work together

\item  He is open to me

\item  She understands me.

\item  I trust him.

\item  He advises me.
\end{enumerate}
\end{enumerate}

\noindent 

\noindent \textbf{RESULTS ANALYSIS:}

\begin{enumerate}
\item \textbf{ }Both male and female had higher tendency to have a best friend than not providing my hypothesis of "everyone has a best friend well, at least most of us".\textbf{}

\item \textbf{ }The female participants with best friends were more free to have best friends of the opposite sex as opposed to the comparable male participants who only had best friends of the same sex.

\item  Both the male and female participants with best friends had higher and equal tendency to have best friends of their peers however, while the male showed potential to have a younger or older best friend, the female were not interested in a younger best friend.

\item  Both male and female participants believe that a best can be a relative more than not.

\item  The reasons for why one considers other person to be the best friend can be summarized as;

\item  "Someone like me, understands me and is willing to share with me."

\item  The ranges of years for which participants have been with their best friends vary from recent to very long.

\item  Participants without best friends considered not to see anything special about their friends.
\end{enumerate}

\noindent 

\noindent \textbf{SOME DERIVED CONCLUSIONS:}

\begin{enumerate}
\item \textbf{ }Best friends are special friends.

\item  A best friend is a person with whom one shares common interests and beliefs.

\item  Most people have best friends.

\item  For someone to be your best friend, it doesn't matter how long you have spent with them but how much you understand them.

\item  Male persons tend not to consider female friends as best friends.

\item  People can consider a relative to be a best friend.
\end{enumerate}

\noindent 

\noindent 

\noindent 

\noindent 


\end{document}

